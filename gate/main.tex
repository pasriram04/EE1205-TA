\documentclass[10pt,-letter paper]{article}
\usepackage[left=1in, right=0.75in, top=1in, bottom=0.75in]{geometry}
\usepackage{graphicx} % Required for inserting images
\usepackage{siunitx}
\usepackage{setspace}
\usepackage{gensymb}
\usepackage{xcolor}
\usepackage{caption}
%\usepackage{subcaption}
\doublespacing
\singlespacing
\usepackage[none]{hyphenat}
\usepackage{amssymb}
\usepackage{relsize}
\usepackage[cmex10]{amsmath}
\usepackage{mathtools}
\usepackage{amsmath}
\usepackage{commath}
\usepackage{amsthm}
\interdisplaylinepenalty=2500
%\savesymbol{iint}
\usepackage{txfonts}
%\restoresymbol{TXF}{iint}
\usepackage{wasysym}
\usepackage{amsthm}
\usepackage{mathrsfs}
\usepackage{txfonts}
\let\vec\mathbf{}
\usepackage{stfloats}
\usepackage{float}
\usepackage{cite}
\usepackage{cases}
\usepackage{subfig}
%\usepackage{xtab}
\usepackage{longtable}
\usepackage{multirow}
%\usepackage{algorithm}
\usepackage{amssymb}
%\usepackage{algpseudocode}
\usepackage{enumitem}
\usepackage{mathtools}
%\usepackage{eenrc}
%\usepackage[framemethod=tikz]{mdframed}
\usepackage{listings}
%\usepackage{listings}
\usepackage[latin1]{inputenc}
%%\usepackage{color}{   
%%\usepackage{lscape}
\usepackage{textcomp}
\usepackage{titling}
\usepackage{hyperref}
%\usepackage{fulbigskip}   
\usepackage{circuitikz}
\usepackage{graphicx}
\lstset{
  frame=single,
  breaklines=true
}
\let\vec\mathbf{}
\usepackage{enumitem}
\usepackage{graphicx}
\usepackage{siunitx}
\let\vec\mathbf{}
\usepackage{enumitem}
\usepackage{graphicx}
\usepackage{enumitem}
\usepackage{tfrupee}
\usepackage{amsmath}
\usepackage{amssymb}
\usepackage{mwe} % for blindtext and example-image-a in example
\usepackage{wrapfig}
\graphicspath{{figs/}}
\providecommand{\cbrak}[1]{\ensuremath{\left\{#1\right\}}}
\providecommand{\brak}[1]{\ensuremath{\left(#1\right)}}
\newcommand{\sgn}{\mathop{\mathrm{sgn}}}
\providecommand{\abs}[1]{\left\vert#1\right\vert}
\providecommand{\res}[1]{\Res\displaylimits_{#1}} 
\providecommand{\norm}[1]{\left\lVert#1\right\rVert}
%\providecommand{\norm}[1]{\lVert#1\rVert}
\providecommand{\mtx}[1]{\mathbf{#1}}
\providecommand{\mean}[1]{E\left[ #1 \right]}
\providecommand{\fourier}{\overset{\mathcal{F}}{ \rightleftharpoons}}
%\providecommand{\hilbert}{\overset{\mathcal{H}}{ \rightleftharpoons}}
\providecommand{\system}{\overset{\mathcal{H}}{ \longleftrightarrow}}
	%\newcommand{\solution}[2]{\textbf{Solution:}{#1}}
%\newcommand{\solution}{\noindent \textbf{Solution: }}
\newcommand{\cosec}{\,\text{cosec}\,}
\providecommand{\dec}[2]{\ensuremath{\overset{#1}{\underset{#2}{\gtrless}}}}
\newcommand{\myvec}[1]{\ensuremath{\begin{pmatrix}#1\end{pmatrix}}}
\newcommand{\myaugvec}[2]{\ensuremath{\begin{amatrix}{#1}#2\end{amatrix}}}
\newcommand{\mydet}[1]{\ensuremath{\begin{vmatrix}#1\end{vmatrix}}}
\date{\today}
\begin{document}
\author{Pandrangi Aditya Sriram}
\title{Gate IN2017, 20}
\date{FWC22249}
\maketitle
\begin{enumerate}
\item $A$ and $B$ are logical inputs and $X$ is the logical output shown in the figure. The output $X$ is related to $A$ and $B$ by
\hfill{(GATE IN 2017)}
\end{enumerate}
\begin{figure}[!ht]
\centering
\begin{circuitikz}
\tikzstyle{every node}=[font=\normalsize]
\draw (7.75,11.5) node[ieeestd not port, anchor=in](port){} (port.out) to[short] (10.5,11.5);
\draw (port.in) to[short] (6.5,11.5);
\draw (7.75,9.5) node[ieeestd not port, anchor=in](port){} (port.out) to[short] (10.25,9.5);
\draw (port.in) to[short] (6.75,9.5);
\draw (11.75,10.75) to[short] (12,10.75);
\draw (11.75,10.25) to[short] (12,10.25);
\draw (12,10.75) node[ieeestd and port, anchor=in 1, scale=0.89](port){} (port.out) to[short] (14,10.5);
\draw (8,7.5) to[short] (8.25,7.5);
\draw (8,7) to[short] (8.25,7);
\draw (8.25,7.5) node[ieeestd and port, anchor=in 1, scale=0.89](port){} (port.out) to[short] (10,7.25);
\draw (14,10.5) to[short] (14,10.5);
\draw (14,10) to[short] (14,10);
\draw (14,10.5) node[ieeestd or port, anchor=in 1, scale=0.89](port){} (port.out) to[short] (15.75,10.25);
\draw[] (6.75,9.5) to[short] (4.25,9.5);
\draw [](5.75,9.5) to[short] (5.75,7);
\draw[] (8,7) to[short] (5.75,7);
\draw[] (8,7.5) to[short] (6.5,7.5);
\draw [](6.5,8) to[short] (6.5,7.5);
\draw[] (11.75,10.75) to[short] (10.5,10.75);
\draw[] (11.75,10.25) to[short] (10.5,10.25);
\draw [](10.5,10.25) to[short] (10.5,9.5);
\draw[] (10.5,9.5) to[short] (10.25,9.5);
\draw [](14,10) to[short] (14,7.5);
\draw [](10,7.25) to[short] (14,7.25);
\draw [](14,7.25) to[short] (14,7.75);
\draw [](10.5,11.5) to[short] (10.5,10.75);
\draw[] (6.5,11.5) to[short] (4.25,11.5);
\draw [](6.5,8) to[short] (6.5,9.25);
\draw [](6.5,11.5) to[short] (6.5,9.75);
\draw [](6.5,10) to[short] (6.5,9.25);
\draw [](4.25,9.5) to[short, -o] (3.75,9.5);
\draw [](4.25,11.5) to[short, -o] (3.75,11.5);
\node [font=\normalsize] at (3.5,11.5) {A};
\draw [](15.75,10.25) to[short, -o] (16.25,10.25);
\node [font=\normalsize] at (3.5,9.5) {B};
\node [font=\normalsize] at (16.5,10.25) {X};
\draw (5.75,9.5) to[short, -*] (5.75,9.5);
\draw (6.5,11.5) to[short, -*] (6.5,11.5);
\end{circuitikz}

\label{fig:block_diagram}
\end{figure}
\begin{enumerate}[label=\Alph*.]
\item $X = \overline{A}B + \overline{B}A$
\item $X = AB + \overline{B}A$
\item $X = AB + (\overline{B})(\overline{A})$
\item $X = (\overline{A})(\overline{B}) + \overline{B}A$
\end{enumerate}
\end{document}