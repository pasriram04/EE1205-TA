\documentclass[12pt,-letter paper]{article}

%\usepackage[left=1.5in,right=1in,top=1in,bottom=1in]{geometry}
%\usepackage[left=1.5in,right=1in]{geometry}
%\usepackage{geometry}
%\makeatletter%
%\textheight     243.5mm
%\textwidth      183.0mm
%\textwidth=31pc%
%\textheight=48pc
\usepackage{lipsum}% this package is included to get dummy paragraphs for sample purpose.
\usepackage{ulem}
\usepackage{alltt}
\usepackage{tfrupee}
\usepackage[anticlockwise,figuresright]{rotating}
\usepackage{pstricks}
\usepackage{wrapfig}
\usepackage{pstcol,pst-grad}
 \usepackage{bm}
\usepackage{enumitem}
\usepackage{listings}
    \usepackage{color}                                            %%
    \usepackage{array}                                            %%
    \usepackage{longtable}                                        %%
    \usepackage{calc}                                             %%
    \usepackage{multirow}                                         %%
    \usepackage{hhline}                                           %%
    \usepackage{ifthen}                                           %%
  %optionally (for landscape tables embedded in another document): %%
    \usepackage{lscape}     
    \usepackage{gensymb}     
    \usepackage{tabularx}
\usepackage{ifthen}%
\usepackage{amsmath}%
\usepackage{color}%
\usepackage{float}%
\usepackage{graphicx}%
%\usepackage[right]{showlabels}%
\usepackage{boites}%
\usepackage{boites_exemples}%
\usepackage{graphicx,pstricks}
%\usepackage{enumerate}%
\usepackage{latexsym}
\usepackage[fleqn]{mathtools}
\usepackage{amssymb}
\usepackage{amssymb,amsfonts,amsthm}
\usepackage{mathrsfs,makeidx,listings,verbatim,moreverb}
%%\usepackage{amsthm,mathrsfs,makeidx,listings,verbatim,moreverb}
%\let\eqref\ref%  updated on 20th April 2017

\usepackage{hyperref}%
%\usepackage[dvips]{hyperref}%
\hypersetup{bookmarksopen=false}%
\usepackage{breakurl}%
\usepackage{tkz-euclide} % loads  TikZ and tkz-base
\DeclarePairedDelimiter\abs{\lvert}{\rvert}

\newcommand{\solution}{\noindent \textbf{Solution: }}
\providecommand{\mbf}{\mathbf}
\providecommand{\rank}{\text{rank}}
%\providecommand{\pr}[1]{\ensuremath{\Pr\left(#1\right)}}
\providecommand{\qfunc}[1]{\ensuremath{Q\left(#1\right)}}
\providecommand{\sbrak}[1]{\ensuremath{{}\left[#1\right]}}
\providecommand{\lsbrak}[1]{\ensuremath{{}\left[#1\right.}}
\providecommand{\rsbrak}[1]{\ensuremath{{}\left.#1\right]}}
\providecommand{\brak}[1]{\ensuremath{\left(#1\right)}}
\providecommand{\lbrak}[1]{\ensuremath{\left(#1\right.}}
\providecommand{\rbrak}[1]{\ensuremath{\left.#1\right)}}
\providecommand{\cbrak}[1]{\ensuremath{\left\{#1\right\}}}
\providecommand{\lcbrak}[1]{\ensuremath{\left\{#1\right.}}
\providecommand{\rcbrak}[1]{\ensuremath{\left.#1\right\}}}
\newenvironment{amatrix}[1]{%
  \left(\begin{array}{@{}*{#1}{c}|c@{}}
}{%
  \end{array}\right)
}
\theoremstyle{remark}
\newtheorem{rem}{Remark}
\newtheorem{theorem}{Theorem}[section]
\newtheorem{problem}{Problem}
\newtheorem{proposition}{Proposition}[section]
\newtheorem{lemma}{Lemma}[section]
\newtheorem{corollary}[theorem]{Corollary}
\newtheorem{example}{Example}[section]
\newtheorem{definition}[problem]{Definition}
\newcommand{\sgn}{\mathop{\mathrm{sgn}}}
%\providecommand{\abs}[1]{\left\vert#1\right\vert}
%\providecommand{\res}[1]{\Res\displaylimits_{#1}} 
%\providecommand{\norm}[1]{\left\lVert#1\right\rVert}
%\providecommand{\norm}[1]{\lVert#1\rVert}
\providecommand{\mtx}[1]{\mathbf{#1}}
%\providecommand{\mean}[1]{E\left[ #1 \right]}
\providecommand{\fourier}{\overset{\mathcal{F}}{ \rightleftharpoons}}
%\providecommand{\hilbert}{\overset{\mathcal{H}}{ \rightleftharpoons}}
\providecommand{\system}{\overset{\mathcal{H}}{ \longleftrightarrow}}
	%\newcommand{\solution}[2]{\textbf{Solution:}{#1}}
%\newcommand{\solution}{\noindent \textbf{Solution: }}
\newcommand{\cosec}{\,\text{cosec}\,}
\providecommand{\dec}[2]{\ensuremath{\overset{#1}{\underset{#2}{\gtrless}}}}
\newcommand{\myvec}[1]{\ensuremath{\begin{pmatrix}#1\end{pmatrix}}}
\newcommand{\myaugvec}[2]{\ensuremath{\begin{amatrix}{#1}#2\end{amatrix}}}
\newcommand{\mydet}[1]{\ensuremath{\begin{vmatrix}#1\end{vmatrix}}}
\newcommand\figref{Fig.~\ref}
\newcommand\appref{Appendix~\ref}
\newcommand\tabref{Table~\ref}
\newcommand{\romanNumeral}[1]{\uppercase\expandafter{\romannumeral#1}}
%\newcommand{\pr}[1]{\mathbb{P}(#1)}
%\numberwithin{equation}{section}
%\numberwithin{equation}{subsection}
%\numberwithin{problem}{section}
%\numberwithin{definition}{section}
%\makeatletter
%\@addtoreset{figure}{problem}
%\makeatother

%\let\StandardTheFigure\thefigure
\let\vec\mathbf
\def\inputGnumericTable{}                                 %%
%New macro definitions
\newcounter{matchleft}\newcounter{matchright}

\newenvironment{matchtabular}{%
  \setcounter{matchleft}{0}%
  \setcounter{matchright}{0}%
  \tabularx{\textwidth}{%
    >{\leavevmode\hbox to 1.5em{\stepcounter{matchleft}\arabic{matchleft}.}}X%
    >{\leavevmode\hbox to 1.5em{\stepcounter{matchright}\alph{matchright})}}X%
    }%
}{\endtabularx}

\title{CBSE Class 12 65/2/S}
\date{\today}
\author{FWC22249: Pandrangi Aditya Sriram}
\begin{document}

\maketitle

\begin{enumerate}
\section{Matrices}
\item If $\vec{A} = \myvec{0 & 3 \\ 2 & -5}$ and $k\vec{A} = \myvec{0 & 4a \\ -8 & 5b}$ find the values of $k$ and $a$.
\item If $\vec{A} = \myvec{1 & 2 \\ 3 & -1}$ and $\vec{B} = \myvec{1 & -4 \\ 3 & -2}$ find $\mydet{\vec{A}\vec{B}}$.
\item If $\vec{A}$ is a square matrix such that $\mydet{\vec{A}} = 5$, write the value of $\mydet{\vec{A}\vec{A}'}$.
\item Ishan wants to donate a rectangular plot of land for a school in his village. When he was asked to give dimensions of the plot, he told that if its length is decreased by $50$ $m$ and breadth is increased by $50$ $m$, then its area will remain same, but if length is decreased by $10$ $m$ and breadth is decreased by $20$ $m$, then its area will decrease by $5300$ $m^{2}$. Using matrices, find the dimensions of the plot. Also give reason why he wants to donate the plot for a school.
\item Using properties of determinant, prove that:
\begin{align*}
\mydet{\brak{b+c}^{2} & a^{2} & bc \\ \brak{c+a}^{2} & b^{2} & ca \\ \brak{a+b}^{2} & c^{2} & ab} = \brak{a-b}\brak{b-c}\brak{c-a}\brak{a+b+c}\brak{a^{2} + b^{2} + c^{2}}
\end{align*}
\item Using elementary row operations, find the inverse of the following matrix: $\myvec{2 & -1 & 3 \\ -5 & 3 & 1 \\ -3 & 2 & 3}$
\section{Probability}
\item There are two bags A and B. Bag A contains $3$ white and $4$ red balls whereas bag B contains $4$ white and $3$ red balls. Three balls are drawn at random (without replacement) from one of the bags and are found to be two white and one red. Find the probability that these were drawn from bag B.
\item Three numbers are selected at random (without replacement) from first six positive integers. If $X$ denotes the smallest of the three numbers obtained, find the probability distribution of $X$. Also find the mean and variance of the distribution.
\section{Algebra}
\item Prove that 
\begin{align*}
2\sin^{-1}\brak{\frac{3}{5}}-\tan^{-1}\brak{\frac{17}{31}} = \frac{\pi}{4}
\end{align*}
\item Solve the equation for x:
\begin{align*}
\cos\brak{\tan^{-1}x} = \sin\brak{\cot^{-1}\frac{3}{4}}
\end{align*}
\section{Vectors}
\item Find the distance between the planes $\overrightarrow{r} \cdot \brak{2\hat{i} - 3\hat{j} + 6\hat{k}} - 4 = 0$ and $\overrightarrow{r} \cdot \brak{6\hat{i} - 9\hat{j} + 18\hat{k}} + 30 = 0$
\item If $\overrightarrow{a}$ and $\overrightarrow{b}$ are unit vectors, then what is the angle between $\overrightarrow{a}$ and $\overrightarrow{b}$ for $\overrightarrow{a} - \sqrt{2}\overrightarrow{b}$ to be a unit vector?
\item If vectors $\overrightarrow{a}$ and $\overrightarrow{b}$ are such that $\mydet{\overrightarrow{a}} = \frac{1}{2}$, $\mydet{\overrightarrow{b}} = \frac{4}{\sqrt{3}}$ and $\mydet{\overrightarrow{a}\times \overrightarrow{b}} = \frac{1}{\sqrt{3}}$, then find $\mydet{\overrightarrow{a} \cdot \overrightarrow{b}}$.
\item Find the equation of the plane passing through the points A $\brak{3, 2, 1}$, B $\brak{4, 2, -2}$ and C $\brak{6, 5, -1}$ and hence find the value of $\lambda$ for which A $\brak{3, 2, 1}$, B $\brak{4, 2, -2}$, C $\brak{6, 5, -1}$ and D $\brak{\lambda, 5, 5}$ are coplanar.
\item Find the co-ordinates of the point where the line $\overrightarrow{r} = \brak{-\hat{i}-2\hat{j}-3\hat{k}} + \lambda \brak{3\hat{i}+4\hat{j}+3\hat{k}}$ meets the plane which is perpendicular to the vector $\overrightarrow{n} = \hat{i} + \hat{j} + 3\hat{k}$ and at a distance of $\frac{4}{\sqrt{11}}$ from origin. 
\item Given that vectors $\overrightarrow{a}$, $\overrightarrow{b}$. $\overrightarrow{c}$ form a triangle such that $\overrightarrow{a} = \overrightarrow{b} + \overrightarrow{c}$. Find $p$, $q$, $r$, $s$ such that area of triangle is $5\sqrt{6}$ where $\overrightarrow{a} = p\hat{i} + q\hat{j} + r\hat{k}$, $\overrightarrow{b} = s\hat{i} + 3\hat{j} + 4\hat{k}$ and $\overrightarrow{c} = 3\hat{i} + \hat{j} - 2\hat{k}$.
\item Find the equation of the plane containing two parallel lines $\frac{x-1}{2} = \frac{y+1}{-1} = \frac{z}{3}$ and $\frac{x}{4} = \frac{y-2}{-2} = \frac{z+1}{6}$. Also, find if the plane thus obtained contains the line $\frac{x - 2}{3} = \frac{y - 1}{1} = \frac{z-2}{5}$ or not.
\section{Functions}
\item Find $k$, if $f(x) = 
    \begin{cases} 
        k \sin \brak{\frac{\pi}{2}\brak{x+1}},& x \leq 0 \\ 
        \frac{\tan x - \sin x}{x^{3}},& x > 0 
    \end{cases}$ is continuous at $x = 0$.
\item Let $f : \mathbb{N} \xrightarrow{} \mathbb{N}$ be a function defined as $f(x) = 4x^{2} + 12x + 15$. Show that $f : \mathbb{N} \xrightarrow{} \mathbb{S}$ is invertible (where $\mathbb{S}$ is range of $f$). Find the inverse of $f$ and hence find $f^{-1}(31)$ and $f^{-1}(87)$
\section{Differentiation}
\item Differentiate $\brak{\sin 2x}^x + \sin^{-1}\brak{\sqrt{3x}}$ with respect to $x$.
\item Differentiate $\tan^{-1}\brak{\frac{\sqrt{1 + x^{2}} - \sqrt{1 - x^{2}}}{\sqrt{1 + x^{2}} + \sqrt{1 - x^{2}}}}$ with respect to $\cos^{-1}x^{2}$
\item Find the equation of normal to the curve $ay^{2} = x^{3}$ at the point whose $x$ coordinate is $am^{2}$.
\item Determine the intervals in which the function $f(x) = x^{4} - 8x^{2} + 22x^{2} - 24x +21$ is strictly increasing or strictly decreasing.
\item Find the maximum and minimum values of $f(x) = \sec x + \log\brak{\cos^{2} x}$, $0 < x < 2\pi$.
\section{Integration}
\item Evaluate: 
\begin{align*}
\int_0^\frac{\pi}{2} \frac{\sin^{2}x}{\sin x + \cos x} dx
\end{align*}
\item Evaluate:
\begin{align*}
\int_0^1 \cot^{-1}\brak{1 - x + x^{2}} dx
\end{align*}
\item Find: 
\begin{align*}
\int \sbrak{\log\brak{\log x} + \frac{1}{\brak{\log x}^{2}}} dx
\end{align*}
\item Find: 
\begin{align*}
\int \frac{1 - \sin x}{\sin x \brak{1 + \sin x}}dx
\end{align*}
\item Solve the differential equation: 
\begin{align*}
2y e^{\frac{x}{y}}dx + \brak{y - 2xe^{\frac{x}{y}}}dy = 0
\end{align*}
\item Solve the differential equation: 
\begin{align*}
\brak{x + 1} \frac{dy}{dx} - y = e^{3x}\brak{x + 1}^{3}
\end{align*}
\item Using integration, find the area of the region 
\begin{align*}
\cbrak{\brak{x,y} : y^{2} \leq 6ax, x^{2} + y^{2} \leq 16a^{2}}
\end{align*}
\section{Optimization}
\item A diet is to contain 80 units of Vitamin A and 100 units of minerals. Two foods $F_1$ and $F_2$ are available, costing \rupee 5 per unit and \rupee 6 per unit, respectively. One unit of food $F_1$ contains 4 units of vitamin A and 3 units of minerals whereas one unit of food $F_2$ contains 3 units of vitamin A and 6 units of minerals. Formulate this as a linear programming problem. Find the minimum cost of diet that consists of mixture of these two foods and also meets minimum nutritional requirement.
\end{enumerate}
\end{document}